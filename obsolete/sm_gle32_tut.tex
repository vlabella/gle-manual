%%%%%%%%%%%%%%%%%%%%%%%%%%%%%%%%%%%%%%%%%%%%%%%%%%%%%%%%%%%%%%%%%%%%%%%%
%                                                                      %
% GLE - Graphics Layout Engine <http://glx.sourceforge.io/>            %
%                                                                      %
% Modified BSD License                                                 %
%                                                                      %
% Copyright (C) 2009 GLE.                                              %
%                                                                      %
% Redistribution and use in source and binary forms, with or without   %
% modification, are permitted provided that the following conditions   %
% are met:                                                             %
%                                                                      %
%    1. Redistributions of source code must retain the above copyright %
% notice, this list of conditions and the following disclaimer.        %
%                                                                      %
%    2. Redistributions in binary form must reproduce the above        %
% copyright notice, this list of conditions and the following          %
% disclaimer in the documentation and/or other materials provided with %
% the distribution.                                                    %
%                                                                      %
%    3. The name of the author may not be used to endorse or promote   %
% products derived from this software without specific prior written   %
% permission.                                                          %
%                                                                      %
% THIS SOFTWARE IS PROVIDED BY THE AUTHOR "AS IS" AND ANY EXPRESS OR   %
% IMPLIED WARRANTIES, INCLUDING, BUT NOT LIMITED TO, THE IMPLIED       %
% WARRANTIES OF MERCHANTABILITY AND FITNESS FOR A PARTICULAR PURPOSE   %
% ARE DISCLAIMED. IN NO EVENT SHALL THE AUTHOR BE LIABLE FOR ANY       %
% DIRECT, INDIRECT, INCIDENTAL, SPECIAL, EXEMPLARY, OR CONSEQUENTIAL   %
% DAMAGES (INCLUDING, BUT NOT LIMITED TO, PROCUREMENT OF SUBSTITUTE    %
% GOODS OR SERVICES; LOSS OF USE, DATA, OR PROFITS; OR BUSINESS        %
% INTERRUPTION) HOWEVER CAUSED AND ON ANY THEORY OF LIABILITY, WHETHER %
% IN CONTRACT, STRICT LIABILITY, OR TORT (INCLUDING NEGLIGENCE OR      %
% OTHERWISE) ARISING IN ANY WAY OUT OF THE USE OF THIS SOFTWARE, EVEN  %
% IF ADVISED OF THE POSSIBILITY OF SUCH DAMAGE.                        %
%                                                                      %
%%%%%%%%%%%%%%%%%%%%%%%%%%%%%%%%%%%%%%%%%%%%%%%%%%%%%%%%%%%%%%%%%%%%%%%%

\section{32bit DOS Version of GLE}
\index{DOS!32bit}
Axel Rohde has compiled a 32bit DOS version of GLE in 1994 
( email: rohde@physik.uni-kiel.d400.de)\\
On any 386 or better machine this version of GLE should 
run without problems, it's main features are these:
1) No 640K memory restrictions. 2) Much faster.\\
\clearpage
Properties of GLE 32:
\begin{verbatim}
 - All programs are running in the 386-protected-mode and therefore there is
   neither a limited 640kB adress-range nor a 64kB segmentation.
 - 32-bit programs are running faster than their 16-bit-counterparts.
 - There exists a multitude of GRX-graphics-drivers, e.g. for
   TSENG ET4000(W32), S3, 8514A, Cirrus Logic GD 542x, Trident 8900,
   Diamond Viper, ATI Ultra, ATI VGA and EGA. These drivers are highly
   configurable and can use flicker-free high resolution modes.
\end{verbatim}
Installation Quick guide:
\begin{verbatim}
    1) FTP the binary distribution
           ftp tui.marc.cri.nz
           cd pub/gle/gle32
           binary
           mget gle32bi*.zip
    2) Unzip them keeping the directory structure
           cd c:\
           pkunzip gle32bi1.zip -d
           pkunzip gle32bi2.zip -d
           pkunzip gle32bi3.zip -d
           pkunzip gle32bi4.zip -d
           pkunzip gle32bi5.zip -d
           pkunzip gle32bi6.zip -d
    3) Edit the batch file which tells gle where to find it's fonts
       and also what sort of graphics card you have.  
           edit setgle32.bat
       (change the disk and directory as appropriate)
    4) Run the batch script
           setgle32
    5) Try out the new version
            gle_vga
    6) Note most of the programs have been renamed to avoid conflicts!!!
\end{verbatim}

To avoid name-conflicts between a 16-bit and a 32-bit version of GLE, 
all the programs and the environment-variables were renamed. 
All GLE-programs have now unix-style names like \verb#gle_ps# (='psgle'-
Postscript-output), \verb#gle_vga# (='cgle' - VGA-Preview) etc. The names 
of the utilities end with '32' -  \verb#manip32#, \verb#contou32# \dots

Restrictions and Bugs:
\begin{description}		
\item[1.] The vector-fonts of the 32-bit-version are NOT compatible to their 
16-bit-counterparts. They may be compatible to fonts that were created 
under other 32-bit operating-systems.
\item[2.] The on-line-help of \verb#gle_vga# is usable but may sometimes look different 
compared to the original.
\item[3.] Makefmt and fbuild are missing. DJGPP's Library lacks ecvt(). 
Both programs are used to calculate vector-fonts in the Unix-version 
from the source-distribution. Both programs are NOT included in the 
16-bit DOS-version, too. This package includes all (already calculated) 
fonts from the Unix-source distribution. They were calculated under Linux, 
a free Unix-implementation for i386-PC's and higher.
\item[4.] The DVI-drivers are not testet! 
\item[5.] Surface (\verb#surf_vga#) hanged under unknown circumstances while loading 
one data-file. \verb#surf_vga# can be stopped by pressing Control-Pause. 
If this happens, load the data-file into an editor, save it, and try it 
again.
\end{description}

In \verb#gle32b.txt# are the more detailed instructions provided by Axel Rohde.
Read these carefully if you have any problems.
