%%%%%%%%%%%%%%%%%%%%%%%%%%%%%%%%%%%%%%%%%%%%%%%%%%%%%%%%%%%%%%%%%%%%%%%%
%                                                                      %
% GLE - Graphics Layout Engine <http://glx.sourceforge.io/>            %
%                                                                      %
% Modified BSD License                                                 %
%                                                                      %
% Copyright (C) 2009 GLE.                                              %
%                                                                      %
% Redistribution and use in source and binary forms, with or without   %
% modification, are permitted provided that the following conditions   %
% are met:                                                             %
%                                                                      %
%    1. Redistributions of source code must retain the above copyright %
% notice, this list of conditions and the following disclaimer.        %
%                                                                      %
%    2. Redistributions in binary form must reproduce the above        %
% copyright notice, this list of conditions and the following          %
% disclaimer in the documentation and/or other materials provided with %
% the distribution.                                                    %
%                                                                      %
%    3. The name of the author may not be used to endorse or promote   %
% products derived from this software without specific prior written   %
% permission.                                                          %
%                                                                      %
% THIS SOFTWARE IS PROVIDED BY THE AUTHOR "AS IS" AND ANY EXPRESS OR   %
% IMPLIED WARRANTIES, INCLUDING, BUT NOT LIMITED TO, THE IMPLIED       %
% WARRANTIES OF MERCHANTABILITY AND FITNESS FOR A PARTICULAR PURPOSE   %
% ARE DISCLAIMED. IN NO EVENT SHALL THE AUTHOR BE LIABLE FOR ANY       %
% DIRECT, INDIRECT, INCIDENTAL, SPECIAL, EXEMPLARY, OR CONSEQUENTIAL   %
% DAMAGES (INCLUDING, BUT NOT LIMITED TO, PROCUREMENT OF SUBSTITUTE    %
% GOODS OR SERVICES; LOSS OF USE, DATA, OR PROFITS; OR BUSINESS        %
% INTERRUPTION) HOWEVER CAUSED AND ON ANY THEORY OF LIABILITY, WHETHER %
% IN CONTRACT, STRICT LIABILITY, OR TORT (INCLUDING NEGLIGENCE OR      %
% OTHERWISE) ARISING IN ANY WAY OUT OF THE USE OF THIS SOFTWARE, EVEN  %
% IF ADVISED OF THE POSSIBILITY OF SUCH DAMAGE.                        %
%                                                                      %
%%%%%%%%%%%%%%%%%%%%%%%%%%%%%%%%%%%%%%%%%%%%%%%%%%%%%%%%%%%%%%%%%%%%%%%%

\begin{center}
\begin{minipage}[t]{11.0cm}
{\scriptsize
\begin{verbatim}
	32 bit DOS version of GLE by Axel Rohde
	---------------------------------------
Introduction: (By Chris)
	For a long time I have been explaining to people that a 32bit 
	dos version of GLE was not possible with currently available 
	compilers and libraries.
	Clearly Axel has no appreciation for the impossible and has gone
	ahead and compiled a version anyway (with absolutely no asistance
	from me).  On any 386 or better machine this version of GLE should 
	run without problems, it's main features are these:
		1) No 640K memory restrictions.
		2) Much faster.

Installation Quick guide:       (By Chris)

	1) FTP the binary distribution
		ftp tui.marc.cri.nz
		cd pub/gle/gle32
		binary
		mget gle32bi*.zip
	2) Unzip them keeping the directory structure
		cd c:\
		pkunzip gle32bi1.zip -d
		pkunzip gle32bi2.zip -d
		pkunzip gle32bi3.zip -d
		pkunzip gle32bi4.zip -d
		pkunzip gle32bi5.zip -d
		pkunzip gle32bi6.zip -d
	3) Edit the batch file which tells gle where to find it's fonts
	and also what sort of graphics card you have.  
		edit setgle32.bat
			(change the disk and directory as appropriate)
	4) Run the batch script
		setgle32
	5) Try out the new version
		gle_vga
	6) Note most of the programs have been renamed to avoid conflicts!!!


	Below are the more detailed instructions provided by Axel Rohde.
	Read these carefully if you have any problems.
\end{verbatim}
}
\end{minipage}
\end{center}
\clearpage
\begin{center}
\begin{minipage}[t]{11.0cm}
{\scriptsize
\begin{verbatim}
                 Installation-manual and documentation 
                     for the 32-Bit DOS version of
                                GLE 3.3 b 

                               by Axel Rohde


 The source-Code of GLE is free and available from many FTP-sites
 (e.g. nic.funet.fi, ftp.informatik.uni-stuttgart.de). GLE 32 is a
 implementation of GLE for MS-DOS. It was compiled with the free port
 of the GNU C-compiler DJGPP by D.J. Delorie. The compiler itself and
 the DJGPP-compiled executables are running with the 32-bit DOS-extender
 GO32 by the same author. Thus, GLE 32 runs only on i386SX and above CPUs.
 GO32 comes with a co-processor-emulator for those without a i387xx
 Real-mode co-prozessor-emulators don't work with GO32. 

 GLE was compiled using Borland-C compatible libraries for text and
 graphics-modes and some filesystem calls.
 There exists for the DJGPP specific graphics-library GRX from Csaba Biegl
 a emulation-library called BCCGRX from Hartmut Schirmer
 to replace calls of the Borland-Graphics-Interface (BGI) with GRX-calls.
 Hartmut has implemented the mouse-functionality, too, with GRX-calls.
 Copyright statements are at the end of this file.

                      Properties of GLE 32

 - All programs are running in the 386-protected-mode and therefore there is
   neither a limited 640kB adress-range nor a 64kB segmentation.
 - 32-bit programs are running faster than their 16-bit-counterparts.
 - There exists a multitude of GRX-graphics-drivers, e.g. for
   TSENG ET4000(W32), S3, 8514A, Cirrus Logic GD 542x, Trident 8900,
   Diamond Viper, ATI Ultra, ATI VGA and EGA. These drivers are highly
   configurable and can use flicker-free high resolution modes.

   
                            Installation

  To understand the next lines, you should have a basic knowledge of
  the MS-DOS operating system and PC-hardware. Take a look into
  your DOS-manual and your hardware documentation in case of doubt.
  
  The DOS-Extender GO32 1.11 has different modes of operation:

  1) VCPI: VCPI is an extension to EMS. EMS can be installed by
     the driver EMM386.EXE of MS-DOS 5.0 and higher.
     The (unoptimized) entry in the system-file CONFIG.SYS looks
     like the follwing lines:
     
        DOS=high,umb
        device=c:\dos\himem.sys
        device=c:\dos\emm386.exe RAM 2048
     
     This entry installs 
     a) 2 MB EMS in the memory area above 1MB,
     b) a page-frame with a size of 64kB between the memory area
        of the graphics-board and the BIOS (upper memory).
     Programs, that use VCPI-calls, can NOT run under Windows or the
     DOS-emulation of OS/2 2.x. Other products, such as QEMM386 can
     also be used.
\end{verbatim}
}
\end{minipage}
\end{center}
\clearpage
\begin{center}
\begin{minipage}[t]{11.0cm}
{\scriptsize
\begin{verbatim}
     2) DPMI: DPMI can be installed with special drivers like QDPMI (along 
     with QEMM 6.0 or higher) or 386ToTheMax 7.x. Both the DOS-emulation 
     of OS/2 2.x and Windows 3.x provide DPMI as a default. The
     DPMI-interface does not allows direct access to (graphics-)hardware.
     Only those programs, that don't use graphic-mode, can run with
     DPMI. I had a lot of total system crashes while running
     some of the text-mode programs of GLE under OS/2 2.1. I recommend
     to run GLE 32 under plain DOS.


   To avoid name-conflicts between a 16-bit and a 32-bit version of GLE,
   all the programs and the environment-variables were renamed.
   All GLE-programs have now unix-style names like 'gle_ps' (='psgle'-
   Postscript-output), 'gle_vga' (='cgle' - VGA-Preview) etc. The names
   of the utilities end with '32' -  'manip32', 'contou32'.... 
  
   In the example on the following lines, the progams are installed in the
   directory d:\gle32, the fonts are in d:\gle32\fonts. GLE 32 searches for
   its vector-fonts in the directory GLE_TOP. Don't forget the trailing slash.
   In addition to this, GL32FONT points to the directory where the bitmap-fonts
   (you can see them in the status-line of the preview) can be found.
   The directory with the DOS-Extender GO32 and, if there's no
   numeric-prozessor installed, the co-prozessor-emulator, must be in the
   path-environment.

     set GLE_TOP=d:/gle32/
     set gle32font=d:/gle32/grxfont
     path=..your normal path..;d:\gle;
     go32=driver d:/gle32/driver/vesa_s3.grn gw 1024 gh 768 tw 80 th 25 nc 256 

   The configuration of graphics-drivers is a little bit more complicated.
   Please study the documentation of the Libraries GRX und BCCGRX and
   the README of GO32 in their directories.
   The environment GO32 sets den path-name and the mode of the driver
   an. This example installs the driver for an S3 graphics-board with a
   resolution of 1024 horizontal 768 vertical pixels and 256 colors.

   There's a second way to install a graphics-mode. If the environment-
   variables GLE32WIDTH, GLE32HEIGHT and GLE32COLORS are set, the graphics-
   mode in the GO32 variable is overridden. You still have to specify a
   driver in the GO32 environment.

       set GLE32WIDTH=800
       set GLE32HEIGHT=600
       set GLE32COLORS=16

   There's a prepared batch-file 'setgle32.bat' to set all the environment-
   variables in the directory gle32.

   To figure out, which modes are supported, try to run the program
   modetest in the directory gle32\driver\doc
                          !!!! WARNING !!!!
   A wrong installed graphics-mode can DESTROY your MONITOR (and/or graphics
   board) if the refresh rate is too high!
                       USE THIS ON YOUR OWN RISK!!!
   You should take a look into the manuals of your monitor and your
   graphics-adapter to figure out, which horizontal frequencies are supported.
   If the horizontal frequency of the monitor is 64 kHz or higher,
   you MAY feel save. (MY Monitor has a horizontal frequency of 64 kHz and
   supports a resolution of 1024x768 with a refresh rate up to 80Hz.)
   If you want to go save, don't set the environment go32 at all!
   This will use the normal (flickering) VGA-mode on a VGA-compatible
   adapter.
\end{verbatim}
}
\end{minipage}
\end{center}
%%\clearpage
\begin{center}
\begin{minipage}[t]{11.0cm}
{\scriptsize
\begin{verbatim}
		       RESTRICTIONS and BUGS
		
   1) The vector-fonts of the 32-bit-version are NOT compatible to their
      16-bit-counterparts. They may be compatible to fonts that were created
      under other 32-bit operating-systems.
   2) The on-line-help of gle_vga is usable but may sometimes look different
      compared to the original.
   3) Makefmt and fbuild are missing. DJGPP's Library lacks ecvt().
      Both programs are used to calculate vector-fonts in the Unix-version
      from the source-distribution. Both programs are NOT included in the
      16-bit DOS-version, too. This package includes all (already calculated)
      fonts from the Unix-source distribution. They were calculated under Linux,
      a free Unix-implementation for i386-PC's and higher. Since a few months,
      there exists an archive with my patches (at the ftp-sites sunsite.unc.edu
      and nic.funet.fi) to compile GLE 3.3b under Linux. 
   4) The DVI-drivers are not testet! (I use Ghostscript or HP2XX for printing
      and converting. Both programs are freely avaiable. Gle_ps and gle_hpgl
      run properly.)
   5) Surface (surf_vga) hanged under unknown circumstances while loading
      one data-file. surf_vga can be stopped by pressing Control-Pause.
      If this happens, load the data-file into an editor, save it, and try it
      again.


		Access to the DJGPP patches for the Source

       I'll release them in the near future after contacting Chris Pugmire.
       You need a fixed version of Hartmut Schirmers BCC2GRX-library
       to (properly) compile gle_vga and surf_vga. This will be released soon. 


			Postscript Documentation
			
       The complete Postscript documentaion is in the directory
       gle32\postscri.doc. I made diff-files to patch them in a
       Ghostscript-printable state. For further information, take
       a look into the file gle32\postscri.doc\readme.pat. The 'patch'
       utility from the 1.11 release of DJGPP is included in this package.


\end{verbatim}
}
\end{minipage}
\end{center}
%%\clearpage
\begin{center}
\begin{minipage}[t]{11.0cm}
{\scriptsize
\begin{verbatim}
                           LEGAL STUFF

Copyright-holders are:

GLE:             unknown to me, take a look into the original
                 documentation in the directory gle32\gle.doc\ and ask
                 Chris Pugmire

DJGPP and GO32:  D. J. Delorie, GNU LIBRARY GENERAL PUBLIC LICENSE and
                 GNU PUBLIC LICENSE,
                 see the files in the directory gle32\go32.doc

GRX:             Csaba Biegl, GNU LIBRARY GENERAL PUBLIC LICENSE,
                 (see gle32\driver\doc\grx.doc)
		  
BCC2GRX:          Hartmut Schirmer, GNU LIBRARY GENERAL PUBLIC LICENSE,
                  (see gle32\driver\doc\bccgrx.doc)

MS-DOS:          Microsoft Corporation (if you haven't guessed...)

DJGPP, GRX and BCC2GRX are available from many ftp-sites. For examples see
the file gle32\driver\doc\bccgrx.doc.


LIKE ANYTHING ELSE THAT'S FREE, GLE 32 AND ITS ASSOCIATED UTILITIES ARE
PROVIDED AS IS AND COME WITH NO WARRANTY OF ANY KIND, EITHER EXPRESSED OR
IMPLIED. IN NO EVENT WILL THE COPYRIGHT HOLDERS BE LIABLE FOR ANY DAMAGES
RESULTING FROM THE USE OF THIS SOFTWARE.
 
These programs are distributed in the hope that it will be useful,
but WITHOUT ANY WARRANTY; without even the implied warranty of
MERCHANTABILITY or FITNESS FOR A PARTICULAR PURPOSE.

Bug-reports are wellcome (to me and Chris Pugmire). It can not be guaranteed
that they can be fixed.


       Axel Rohde, 19.1.94,    email: rohde@physik.uni-kiel.d400.de

       Chris Pugmire's email address: chrisp@marc.cri.nz 
\end{verbatim}
}
\hfill {\small gle32b.txt}\\
\end{minipage}
\end{center}
